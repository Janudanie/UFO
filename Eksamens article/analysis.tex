\section{Analysis}
\label{section:analysis}
The analysis will be composed from a range of articles and write-ups by authors who argue for and against TDD or TLD. 
Some will also be more objective views and focus more on the comparison of the two, and when one excels over the other. 
Some of the articles will be subjective with the purpose of getting you to subscribe to one strategy. 
Then build a conclusion from their citations and experience.   



\subsection{Why TDD?}
Code becomes modular:
When using TDD, the code becomes more modular. The reason for this is that a developer writes a test and then writes the code to satisfy the test.  This makes the developer think more in modules as they are easily tested. The real downside here, is that the developer is going to use more time writing the code.

Detect bugs early:
When writing the test first, the developers get instant feedback if the code is working as it should, by simply running the test. This will help with fiding and fixing bugs earlier.  

Cheaper to fix:
When a developer can get instant results on whether the code has passed or not, the developer will quickly change the code to be correct, while his mind is focused on this piece of code. If the developer comes back to test the code even a few days later, he will first have to get the code into his head, which takes a lot more time.
A drawback can be, that the test suites becomes to big and therefor takes too long to execute. Then there is a lot of time needed to be invested into reducing the test suite.

Refactoring becomes easier:
When a developer has well defined tests in place, refactoring code is remarkably simple, as the developer would get instant results on whether this refactored code has passed or not. Writing a well-defined test however, can take a long time, and in the most extremes, the tests for the code can take more time than writing the code itself.

Faster development:
When a developer is writing some code and want to test it, if they are not using TDD, they would have to manually do the tests. This could take a lot of time. But if there are automatic tests, this can be done in a fraction of the time. If, however functionality of the code is changed rapidly or functions are added just to be removed later, the development would suffer. Because test suites for the function would still have to be made first.

Cleaner Code:
Starting to use TDD can be hard, because it changes the way of thinking when it comes to writing code. This will be hard and take a lot of effort. Overtime however, the more experience the developer gets, the easier it becomes, and the code would be better and more clearer on the first pass at writing the code.

\subsection{TLD}

TLD is a very different from TDD. Test Last development does not mean to test at the end of the total application.
But it entails that the developer execute some unite tests against the code when the developed unit is done. 
All in all does it mean that all testing is done before code goes into Production. But not automated.~\cite{gm2011}\newline



TLD Pros and Cons
It takes less time than TDD. TDD takes up to 16\% longer than TLD.
~\cite{bl2003} This is due of the high learning curve of TDD. 
A lot of time in TDD is used on skipping between code and tests and focusing on writing failing tests. 
Using TLD the code base becomes a lot smaller, and more linear and easy to understand, 
with simpler code and less Cylomatic complexity. 
The developer has to spend less time refactoring as the code is written to solve the problem and move on to the next specification. 
Some of the things TLD lacks is that it has up to 52\% less test than TDD, 
due to TDD’s aggressive testing. This leads to more bugs in a TLD system. 
The code also tent to become “uglier” and more solidified, making it hard to change later. 
Lastly, time is money, and TLD requires a lot more time and effort to maintain later.~\cite{ss2014}\newline

When skimming the pros and cons, we get the idea that, 
while TLD is easier, the quality is inferior to TDD. 
Higher quality code is one of the pros of TDD often mentioned. 
In 2005 by Lech Madeyski made a study on the correlation between TDD, 
TLD and code quality. ~\cite{lm2005}. The study has 108 developers, 
spread across experience and coding language. 
They experiment runs over eight coding sessions, the participants are divided into 4 groups.
\begin{itemize}
    \item Solo programming TLD
    \item Solo programming TDD
    \item Pair programming TLD
    \item Pair programming TDD
\end{itemize}

After the eight sessions, the code is inspected and graded on quality. 
Lech Madeyski found that the code quality (Measured by acceptance tests passed) 
was highly affected on which approach was used. 
What is interesting though was that he found that Solo programmers who used TLD had a better 
code quality that Solo programmers using TDD. He also found that there was no difference in code 
quality from the Pair programming groups. This gives a picture, that it is cheaper and better to have a 
single developer during TLD, than two developer using TDD. ~\cite{lm2005}.



The medium article by Simon Redmond~\cite{sr2019}, 
is a more anecdotal and less scientific source, echoes part of Lech Madeyski’s conclusion.
Redmon believes that using TDD over TLD is a costly effort for very little value. 
And talks about how aggressive testing for bugs, and writing code to satisfy code. 
Is time taken away from innovative code and the actual code that progress your product. 
Possible bugs and errors will be handled later when TLD, just like they would be caught in TDD. 
Something that can be recognized in the experiment. \newline

Each strategy has its strength and weakness, 
but the learning curve of the two approaches is one of the big 
factors when choosing. If you have a more experienced developer team, that has the discipline and overview to write a test first, then write the method. Then the team should consider TDD over TLD. If the developer team has less experience, and are more of the “I need this to work now” kind of developer, 
the TLD approach should be considered.  
\subsection{TDD vs TLD - Pros and cons} 

Each of the two test strategies has their pros and cons. Some of the most common pros and cons can be seen 
in the table
\ref{table: procons1} and  \ref{table: procons2}
\begin{center}
\begin{table}[h]    
  
    \begin{tabular}{|l|l|}       
        \hline
        \multicolumn{2}{|c|}{TDD} \\
        \hline
        \textbf{Pros} & \textbf{Cons}  \\ \hline
        Code becomes modular &  Takes a lot of extra coding  \\ 
        Decect bugs early & Can result in major changes to the code later   \\ 
        Cheaper to fix &  Require often reconfiguration of suites  \\
        Refactorring becomes easier &  Hard to make well defined tests  \\      
        Faster development &  A lot of wasted time in constant changing code \\ 
        Cleaner code &  Steep learning curve  \\   
        \hline
    \end{tabular}
    \caption{} 
    \label{table: procons1}
\end{table} 
  
\begin{table}[h] 
      \begin{tabular}{|l|l|}
        \hline
        \multicolumn{2}{|c|}{TLD} \\
        \hline
        \textbf{Pros} & \textbf{Cons}  \\ \hline
        Easy to do and understand &  Expensive to maintaine  \\ 
        Code is ls less complex & Testing can become too wide  \\ 
        Does not require extra refractoring &  Bugs are discovered late in development  \\
        Has a lot less code & No pre-testing of new functions  \\  
        & Risk of missing new bugs  \\  \hline
    
    \end{tabular}
    \caption{}  
    \label{table: procons2}
    \end{table}
\end{center}





