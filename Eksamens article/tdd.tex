\subsection{Why TDD?}
What makes TDD a good choice? This section is going to present a few pro’s when it comes to TDD. These pros are not all the pros found, but some of the most important. They are found in the various articles. ~\cite{ad2020} ~\cite{mf2017} ~\cite{sl2018} ~\cite{various2019}


\begin{itemize}
\item Code becomes modular:

When using TDD, the code becomes more modular. The reason for this is that a developer writes a test and then writes the code to satisfy the test.  This makes the developer think more in modules as they are easily tested. The real downside here, is that the developer is going to use more time writing the code.

\item Detect bugs early:

When writing the test first, the developers get instant feedback if the code is working as it should, by simply running the test. This will help with fiding and fixing bugs earlier.  

\item Cheaper to fix:

When a developer can get instant results on whether the code has passed or not, the developer will quickly change the code to be correct, while his mind is focused on this piece of code. If the developer comes back to test the code even a few days later, he will first have to get the code into his head, which takes a lot more time.
A drawback can be, that the test suites becomes to big and therefor takes too long to execute. Then there is a lot of time needed to be invested into reducing the test suite.

\item Refactoring becomes easier:

When a developer has well defined tests in place, refactoring code is remarkably simple, as the developer would get instant results on whether this refactored code has passed or not. Writing a well-defined test however, can take a long time, and in the most extremes, the tests for the code can take more time than writing the code itself.

\item Faster development:

When a developer is writing some code and want to test it, if they are not using TDD, they would have to manually do the tests. This could take a lot of time. But if there are automatic tests, this can be done in a fraction of the time. If, however functionality of the code is changed rapidly or functions are added just to be removed later, the development would suffer. Because test suites for the function would still have to be made first.

\item Cleaner Code:

Starting to use TDD can be hard, because it changes the way of thinking when it comes to writing code. This will be hard and take a lot of effort. Overtime however, the more experience the developer gets, the easier it becomes, and the code would be better and more clearer on the first pass at writing the code.
\end{itemize}