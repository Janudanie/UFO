\documentclass{report}
\usepackage{url}


\begin{document}


\chapter{krav til et bachelorprojekt}
Denne rapport, skal laves for at dokumentere hvad kravene er til et bachelor projekt.
Rapporten skal skrives i \LaTeX.

\section{Sideantal}
Har ikke kunne finde et godt dokument for at skrive om et side antal.
Jeg har kunne finde en kickoff pdf for tidligere bachelor projekter~\cite{kickoff} der beskriver et side antal der hedder 40 sider + 20 per deltager.
så i grupper på 1-4 hedder det sig 60 til 120 siders rapport. Derudover beskriver den også et minimums side antal på ca 2/3 af side antal, for at rapporten ikke skal virke for kort.

I studieordningen~\cite{studie} beskrives at en side er 2400 anslag, inklusiv mellemrum

\section{Formulering og stav evne}
I studieordningen~\cite{studie} er der beskrevet at det fagligeindhold vægter tungest, men at formulering og stave evne også er vægtet.

\section{Formålet med bachelor projekt}
I studieordningen~\cite{studie} er der skrevet præcist hvad formålet med bachelor eksamens formål er, og hvad du som elev skal vise. Her kan blandt andet nævnes:


\begin{itemize}
  \item Viden.
	\begin{itemize}
	\item den strategiske rolle af test i systemudvikling.
	\item globalisering af softwareproduktion.
	\item systemarkitektur og forståelse af dens strategiske betydning for virksomhedens forretning.
	\item anvendt teori og metode samt udbredte teknologier inden for domænet.
	\item forskellige databasetyper og deres anvendelse.
	\end{itemize}
  \item Færdigheder.
	\begin{itemize}
	\item integrere it-systemer og udvikle systemer, som understøtter fremtidig integration.
	\item anvende kontrakter som en styrings- og koordineringsmekanisme i udviklingsprocessen.
	\item vurdere og vælge databasesystemer, samt designe, redesigne og driftsoptimere databaser.
	\item planlægge og styre udviklingsforløb med mange geografisk adskilte projektdeltagere.
	\item håndtere planlægning og gennemførelse af test af større it-systemer
	\end{itemize}
  \item Kompetencer.
	\begin{itemize}
	\item identificere sammenhænge mellem anvendt teori, metode og teknologi og kan reflektere over disses egnethed i forskellige situationer.
	\item indgå professionelt i samarbejde omkring udvikling af store systemer ved anvendelse af udbredte metoder og teknologier.
	\item sætte sig ind i nye teknologier og standarder til håndtering af integration mellem systemer.
	\item gennem praksis udvikle egen kompetenceprofil fra primært at være en backend-udviklerprofil til at varetage opgaver som systemarkitekt.
	\item håndtere fastlæggelse og realisering af en såvel forretningsmæssig som teknologisk hensigtsmæssig arkitektur for store systemer.
	\end{itemize}
\end{itemize}

\bibliographystyle{plain}
\bibliography{biblo}
\end{document}