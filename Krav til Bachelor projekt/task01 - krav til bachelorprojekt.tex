\documentclass{article}
\usepackage{url}

\title{Krav til en professionsbachelor i softwareudvikling afsluttende rapport}
\date{2020\\ October}
\author{Alex Langhoff}


\begin{document}
\maketitle

Denne artikel vil gennemgå, og komme med anbefalinger til at skrive en professionsbachelor rapport til softwareudvikling.
Denne artikel er skrevet vedhjælp af \LaTeX.


\section{Formålet med rapporten}
Bachelorprojekt på 10, 15 eller 20 ECTS-point, der skal dokumentere den studerendes forståelse af og evne til at reflektere over professionens praksis og anvendelse af teori og metode i relation til en praksisnær problemstilling. Problemstillingen, der skal være central for uddannelsen og professionen, formuleres af den studerende, eventuelt i samarbejde med en privat eller offentlig virksomhed. Institutionen godkender problemstillingen.~\cite{bekendt}


\section{Omfang af rapporten}
Omfanget af rapporten er på 40 sider + 20 sider per deltager~\cite{kickoff}, max 4 deltager:

\begin{itemize}
\item 1 deltager, max 60 sider
\item 2 deltager, max 80 sider
\item 3 deltager, max 100 sider
\item 4 deltager, max 120 sider
\end{itemize}

I studieordningen~\cite{studie} beskrives at en side er 2400 anslag, inklusiv mellemrum.
For at rapporten ikke kommer til at virke for kort, anbefales det at skrive et minimum af 2/3 dele af max sideantal~\cite{kickoff}.
Det skal bemærkes at det faglige indhold vægter tungest, men stave- og formuleringsevne også vægtes~\cite{studie}.



\section{Prøveformen}
Selve prøven er en mundligt prøve over 30 minutter som har denne opbygning:
\begin{itemize}
\item 10 minutter: Fremlæggelse af rapport.
\item 15 minutter: Spørgsmål fra lære og censor.
\item 5 minutter: voteringen.
\end{itemize}

Dette er ikke underbygget af nogle beviser, men er mere hvordan de andre afgangs eksamer er afholdt.
Det på ligger normalt skolen at opbygge eksamen, og melde ud til den studerende.

\section{Formålet med bachelor projekt}
I studieordningen~\cite{studie} er der skrevet præcist hvad formålet med bachelor eksamens formål er, og hvad du som elev skal vise. Her kan blandt andet nævnes:


\begin{itemize}
  \item Viden.
	\begin{itemize}
	\item den strategiske rolle af test i systemudvikling.
	\item globalisering af softwareproduktion.
	\item systemarkitektur og forståelse af dens strategiske betydning for virksomhedens forretning.
	\item anvendt teori og metode samt udbredte teknologier inden for domænet.
	\item forskellige databasetyper og deres anvendelse.
	\end{itemize}
  \item Færdigheder.
	\begin{itemize}
	\item integrere it-systemer og udvikle systemer, som understøtter fremtidig integration.
	\item anvende kontrakter som en styrings- og koordineringsmekanisme i udviklingsprocessen.
	\item vurdere og vælge databasesystemer, samt designe, redesigne og driftsoptimere databaser.
	\item planlægge og styre udviklingsforløb med mange geografisk adskilte projektdeltagere.
	\item håndtere planlægning og gennemførelse af test af større it-systemer
	\end{itemize}
  \item Kompetencer.
	\begin{itemize}
	\item identificere sammenhænge mellem anvendt teori, metode og teknologi og kan reflektere over disses egnethed i forskellige situationer.
	\item indgå professionelt i samarbejde omkring udvikling af store systemer ved anvendelse af udbredte metoder og teknologier.
	\item sætte sig ind i nye teknologier og standarder til håndtering af integration mellem systemer.
	\item gennem praksis udvikle egen kompetenceprofil fra primært at være en backend-udviklerprofil til at varetage opgaver som systemarkitekt.
	\item håndtere fastlæggelse og realisering af en såvel forretningsmæssig som teknologisk hensigtsmæssig arkitektur for store systemer.
	\end{itemize}
\end{itemize}

\bibliographystyle{plain}
\bibliography{biblo}
\end{document}